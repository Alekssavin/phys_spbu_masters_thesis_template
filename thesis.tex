% !TEX program = xelatex
%thesis.tex
%Это основной tex-файл, собирающий документ воедино.
%Необходимые пакеты и команды:
\documentclass[a4paper,14pt]{extarticle} %Размер кегля можно менять в пределах 12-14. Пакет extarticle Поддерживает 14pt.
% \usepackage[utf8]{inputenc} %кодировка.
\usepackage[english,russian]{babel} %языковые пакеты.
\usepackage{fontspec} %шрифты. 
\defaultfontfeatures{Ligatures={TeX},Renderer=Basic}  %служебная команда для XuLaTeX 
\setmainfont[Ligatures={TeX,Historic}]{Times New Roman} %выбор шрифта Times New Roman.
\usepackage[backend=biber,style=gost-numeric,sorting=none, language=auto,hyperref=auto, autolang=other]{biblatex} %для библиографии выбирается стиль ГОСТ. Backend biber -- программа, отображающая библиографию. Она должна быть выставлена в настройках редактора как основная. Ищите в настройках "BibTeX Engine".
\addbibresource{bibliography.bib} %файл с библиографическими данными.
\usepackage{graphicx} %пакет, позволяющий вставлять изображения в документ.о
\graphicspath{{./images/}}  %все изображения хранятся в папке images/
\usepackage{float}  %пакет, позволяющий управлять расположение плавающих объектов, например изображений.
\usepackage{geometry} %пакет, определяющий отступы по краям страниц.
\geometry{top=2cm}  %отступ сверху 20мм.
\geometry{bottom=2cm} %отступ снизу 20мм.
\geometry{left=3cm} %отступ слева 30мм.
\geometry{right=1cm}  %отступ справа 10мм.
\usepackage{setspace}  %полуторный межстрочный интервал.
\onehalfspacing
\usepackage{amsthm,amssymb, amsmath} %AMS стиль для формул.
\usepackage{booktabs} %расширенные возможности редактирования таблиц.
\usepackage{multirow} %объединение нескольких строк в таблицах.
\providecommand{\keywordsru}[1]{\textbf{\textit{Ключевые слова:}} #1}
\providecommand{\keywordsen}[1]{\textbf{\textit{Keywords:}} #1}
%Дополнительные пакеты и команды:
\usepackage{siunitx}  %физические величины.
\usepackage{enumitem} %улучшенные нумерованные списки.
\usepackage{csquotes} %цитаты

% Начало документа:
\begin{document}
  
  \thispagestyle{empty}
\begin{center}
 Федеральное государственное бюджетное образовательное учреждение высшего образования «Санкт-Петербургский государственный университет»\\
  \vspace{6cm}
  \textbf{ИВАНОВ Иван Иванович}\\
  \textbf{Выпускная квалификационная работа}\\
  \textbf{Название выпускной квалификационной работы}\\
  \vspace{1.5cm}
  Уровень образования: магистратура\\
  Направление 03.04.02 «Физика»\\
  Основная образовательная программа «Физика»\\
  Профиль «Физика твёрдого тела»\\
  \vspace{1.5cm}
\end{center}
% \hspace{9cm} %пустое место слева.
\begin{flushright}
  Научный руководитель: \\
  профессор кафедры ФТТ \\
  физический факультет, СПбГУ\\
  д.ф.-м.н. Васильев Василий Васильевич\\
  \vspace{1cm}
  Рецензент:  \\
  старший научный сотрудник кафедры ФТТ \\
  название факультета и института,  \\
  к.ф.-м.н. Петров Пётр Петрович
\end{flushright}
\vfill
\begin{center}
  Санкт-Петербург\\
  2021
\end{center}
 %вставить титульный лист.

  \newpage

  \tableofcontents

  \newpage

  \begin{abstract} %аннотация на русском языке.
  Здесь вставить вашу анатацию на русском языке.Аннотация должна отражать цель исследования, основное содержание и новизну статьи в сравнении с другими, родственными по тематике и целевому назначению, а также полученные результаты.
  Рекомендуемый средний объем аннотации – 600 печатных знаков (ГОСТ 7.0.99-2018). 
\end{abstract}

\keywordsru{ключевые, слова, через, запятую}  %ключевые слова на русском языке.

\selectlanguage{english}  %переключаем язык на английский.

\begin{abstract} %аннотация на английском языке.
  Here is the translation of your abstract into English.
\end{abstract}

\keywordsen{key, words, with, commas} %ключевые слова на английском языке.

\selectlanguage{russian}  %переключаем язык на русский.
  %вставить аннотацию и ключевые слова.

  \newpage

  \section{Введение.}
Здесь можно описать постановку задачи и почему она важна.
 %вставить введение.

  \newpage

  \section{Обзор литературы и предыдущих работ.}
Здесь вставить описание исследований и литературы. Обязательно снабдить ссылками на статьи, например статья Энштейна~\cite{einstein1905} и книга Ильина~\cite{ilyin1998}.
 %вставить обзор литературы и предшествующих работ.

  \newpage

  \section{Основные результаты работы}

Здесь указываем основные результаты работы.

Можем вставить формулу:

\begin{equation}
  \lim_{n\rightarrow\infty} \sum_{i=1}^{N} \Delta x \cdot f(x_i) = \int_a^b f(x) dx
  \label{equationref}
\end{equation}

Можно сослаться на формулу~\ref{equationref}.

Можем вставить рисунок:

\begin{figure}[H]
 \centering %выравнивание по центру
  \includegraphics[width=0.7\textwidth]{pic01.png}
  \caption{Это пример изображения шириной 0.7 размера строки.}
  \label{pictureref}  %название изображения для ссылок.
\end{figure}

Можем сослаться на рисунок~\ref{pictureref}.

Можем вставить таблицу:

\begin{table}[H]
  \centering
  \begin{tabular}{|c|c|c|}
   \hline
   Колонка 1  & Колонка 2 & Колонка 3 \\
   \hline
   1  & 2 & 3 \\
   \hline
  \end{tabular}
  \caption{Это пример таблицы.}
  \label{tableref}  %название таблицы для ссылок.
\end{table}

Можем сослаться на таблицу~\ref{tableref}.

Можем написать текст и сделать сноску внузу страницы\footnote{Сноска и её текст.}.
 %вставить основную часть. Её можно при желании разбит на несколько отдельных документов.

  \newpage

  \section{Заключение}

Здесь необходимо сделать заключение: кратко указать, какие результаты достигнуты и почему они являются ценными.
 %вставить заключительную часть.

  \newpage

  \section{Перечень использованного оборудования, в том числе оборудования  Научного парка СПбГУ}

\begin{enumerate}
  \item Гибридный кластер Нybrid Ресурсного центра «Вычислительный центр» СПбГУ.
\end{enumerate}
  %перечень использованного оборудования.

  \printbibliography[heading=bibintoc] % библиография.

\end{document}
